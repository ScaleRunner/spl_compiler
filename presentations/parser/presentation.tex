\documentclass{beamer}

\usepackage[british]{babel}
\usepackage{graphicx,hyperref,ru,url}

% The title of the presentation:
%  - first a short version which is visible at the bottom of each slide;
%  - second the full title shown on the title slide;
\title[Compiler Construction - Parsers]{Compiler Construction - Parsers}

% Optional: a subtitle to be dispalyed on the title slide
\subtitle{The Road So Far}

% The author(s) of the presentation:
%  - again first a short version to be displayed at the bottom;
%  - next the full list of authors, which may include contact information;
\author[D. Verheijen, A. Andrade]{Dennis Verheijden \inst{1} \and Alan Andrade \inst{2}}

% The institute:
%  - to start the name of the university as displayed on the top of each slide
%    this can be adjusted such that you can also create a Dutch version
%  - next the institute information as displayed on the title slide
\institute[Radboud University Nijmegen]{
    \inst{1} Data Science \inst{2} Software Science \\
    Radboud University Nijmegen}

% Add a date and possibly the name of the event to the slides
%  - again first a short version to be shown at the bottom of each slide
%  - second the full date and event name for the title slide
\date[Parser Presentation]{
    Parser Presentation\\
    15th March 2018}

\begin{document}

\begin{frame}
    \titlepage
\end{frame}

\begin{frame}
    \frametitle{Outline}
    \tableofcontents
\end{frame}

% Section titles are shown in at the top of the slides with the current section 
% highlighted. Note that the number of sections determines the size of the top 
% bar, and hence the university name and logo. If you do not add any sections 
% they will not be visible.
\section{Parser Specifications}

\begin{frame}
    \frametitle{Parser Specifications}
    
    \begin{itemize}
        \item Top-Down Parser
        \item Split up in a Scanner and Parser
        \item Written in Java, so Recursive Descent
        \item Pivoted later to a Top-Down Operator Precedence Parser (more later).
    \end{itemize}
\end{frame}

\section{Encountered Problems}
\begin{frame}
    \frametitle{Left Recursion}
    
        \begin{block}{Left Recursion}
        \begin{align*}
            \text{Field} &= [ \text{Field} (`.` `hd` | `.` `tl` | `.` `fst` | `.` `snd`)]
        \end{align*}
        \end{block}
        \begin{block}{Fixed}
            New Grammar
        \end{block}
\end{frame}


\begin{frame}
    \frametitle{Associativity}
    Changing the grammar resulted in the following:
    \begin{align*}
    a * b * c * d &\rightarrow (a * (b * (c * d))) \\
                  &\rightarrow (((a * b) * c) * d)
    \end{align*}
  
    So we should change the grammar again... OR \textbf{Pratt Parser}
\end{frame}

\section{Pratt Parser}

\begin{frame}
    \frametitle{Pratt Parser}

    \begin{itemize}
        \item Improved Recursive Descent Parser (Vaughan Pratt, 1973)
        \item Associate semantics with tokens instead of grammar rules.
        \item Rewrite grammar rules to individual Parselets.
        \item Operators now have a precedence value.
        \item Complete Grammar Rewrite was needed.
    \end{itemize}
\end{frame}

\begin{frame}
    \frametitle{Example}
    
\end{frame}


\section{Results}

\begin{frame}
    \frametitle{Tests}
    Fun Examples Here
\end{frame}

\section{Conclusion}

\begin{frame}
    \frametitle{Journey So Far}
    
    \begin{itemize}
        \item Worked on the project side-by-side for ~30 hours
        \item Creating a parser by hand is tedious without the use of tools
        \item In hindsight, we were maybe better off using functional languages, or even better hybrid languages like F\#.
        \item 2.5k lines of code (40\% Test)
    \end{itemize}
\end{frame}

\end{document}
